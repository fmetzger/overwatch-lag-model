%!TEX root = paper.tex
%%%%%%%%%%%%%%%%%%%%%%%%%%%%%%%%%%%%%%%%%%%%%%%%%%%%%%%%%%%%%%%%%%%%%%%%%%%%%%%%
\section{Conclusion}
\label{sec:conclusion}

Repeat again, observations for one specific computer / resource environmant, might be hard to generalize. But can still be indicative of general network issues in such environments, because such processes might be tied to other processes, e.g., the renderer, which can not keep their intended output rates.

This lag model for \textsc{Overwatch} is highly customized and unfortunately not a one-size-fits-all solution for every game. But it still gives much needed insights into the dynamics of video games and the deviations from the theoretical properties that are actually happening in praxis. With that being said, even the refined simulation clearly demonstrates the strong influence of the game's frame rate and input and messaging processes investigated here on the \gls{E2E} lag, despite operating under near-ideal network conditions.