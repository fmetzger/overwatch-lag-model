%!TEX root = paper.tex
%%%%%%%%%%%%%%%%%%%%%%%%%%%%%%%%%%%%%%%%%%%%%%%%%%%%%%%%%%%%%%%%%%%%%%%%%%%%%%%%
\section{Conclusion}
\label{sec:conclusion}

This lag model for \textsc{Overwatch} is highly customized to a specific case of resource limitation and unfortunately not a one-size-fits-all solution for every game. But it is indicative of issues in such environments, because components relevant to the lag might negatively interact with other processes, e.g., the renderer, which can not keep its intended output rate, and thus causes the game's networking process to misbehave. Therefore, it gives much needed insight into the dynamics of video games and the deviations from the theoretical properties that are actually occurring in praxis.  This specialized simulation demonstrates quite well the influence of the game's frame rate and input and messaging processes investigated here on the \gls{E2E} lag, despite operating under near-ideal network conditions.