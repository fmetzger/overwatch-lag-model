%!TEX root = paper.tex
%%%%%%%%%%%%%%%%%%%%%%%%%%%%%%%%%%%%%%%%%%%%%%%%%%%%%%%%%%%%%%%%%%%%%%%%
%%%%%%%

\section{Introduction}

In theory, video games are simple: Retrieve an input, process it and update the game state, then display the results. But in praxis this gets a lot more complicated through the dynamic behavior of all involved factors. Especially when you add competitive multiplayer in to the mix and must synchronize the game states of the (authoritative) game server and each of the clients. Each of these components can then contribute to the \gls{E2E} lag, which will have an impact on each players' interactivity, and such on the perceived quality of the game as well \cite{Claypool:2006:LPA:1167838.1167860,7965676}. 

As to the extent of the impact of this lag on the perceived quality, many different factors can play a role, not just, as originally thought, the genre of the game in question \cite{mollertowards,Slivar:2016:CGQ:2910017.2910602}. In theory, the lag components of a video game can be measured \cite{7148095} or modeled in simulation \cite{Metzger+2016}. But in praxis the involved processes can get a lot more dynamic than stated in basic theory. The resource-constrained nature, where the video game can not maintain a stable frame rate, can play a major role. Most research in the past has simply assumed ideal conditions on this matter. Sufficient CPU and GPU resources to maintain, e.g., a frame rate of \SI{60}{\hertz}, as well as sufficient server resources to handle all of the users' input events. Yet, in reality this is often not the case. According to the Steam Hardware Survey\footnote{\url{https://store.steampowered.com/hwsurvey/}} the typical PC video game player tends not to have the latest or most powerful iteration of hardware components. But she still wants to turn up the game's settings to enjoy a better graphical fidelity, often at the cost of frame rate. This resource-constrainedness affects all associated components relevant to the \gls{E2E} lag, including the command send rate from the game client to the server. Similar variations will occur at the server's side and impact the game state update transmission rates to the game clients. It should be noted, that there are many anecdotes of competitive players doing the exact opposite and reduce graphical details to achieve both less visual distraction and better performance.

This paper explores such a resource-starved case and its effect on the popular team-based first-person shooter \textsc{Overwatch} on the PC. We recorded several game sessions and on this basis examine and model all factors that influence the lag. These models are then used to refine the \gls{E2E} lag simulation of \cite{Metzger+2016}, of which this work is a direct follow-up. By their very nature, the results are limited to this specific case and can not necessarily be easily generalized. And yet, the conditions in this experiment are not particularly rare. The game's ideal behavior is just that. Ideal, intended behavior in an environment where everything works perfectly. But due to the large variety in computer hardware and the fact that games often push the given hardware to their limits --- to produce good image quality at acceptable performance levels --- video games often run into resource limits. And this what makes the study of such corner cases nonetheless worthwhile.


% Anectdotally correct, but do we have any evidence, references on this?
% Or do we just argue that this is a common preference, as also targeted at by console manufactures/developers (aim for 30fps instead of something higher)
% Additionally: competitive players will often do the opposite to gain potential (and again, anectdotal) advantages. esp. e.g. observed in CSGO; advantages through both a better performance, but also through less visual complexity, less noise to distract
