%!TEX root = paper.tex
%%%%%%%%%%%%%%%%%%%%%%%%%%%%%%%%%%%%%%%%%%%%%%%%%%%%%%%%%%%%%%%%%%%%%%%%
%%%%%%%

\section{Introduction}

In theory, video games are simple: Retrieve an input, process it to update the game state, then display the results. But in praxis this gets a lot more complicated through the dynamic behavior of all involved factors. Especially when you add competitive multiplayer in to the mix and must synchronize the game states of the (authoritative) game server and each of the clients. Each of these components can then contributed to the \gls{E2E} lag, which will have an impact on each players' interactivity and such on the perceived quality of the game \cite{Claypool:2006:LPA:1167838.1167860,7965676}. As to the extent of the impact of this lag on the perceived quality, many different factors can play a role, not just, as originally thought, the genre of the game in question \cite{mollertowards}. In theory, the lag components of a video game can be rather easily modeled in simulation, as has already been conducted in past studies \cite{Metzger+2016}. But in praxis the involved processes can get a lot more dynamic than stated in basic theory. The resource-constrained nature, where the video game can not maintain a stable frame rate, can play a major role. Most research in the past has assumed ideal conditions on this matter. Sufficient CPU and GPU resources to maintain, e.g., a frame rate of \SI{60}{\hertz}, as well as sufficient resources at the server to handle all all of the users' input events. Yet, in reality this is often not the case. Players have older computers and additionally turn up the game's settings to enjoy a better graphical fidelity at the cost of frame rate. And this in turn affects all associated components relevant to the \gls{E2E} lag, including the command send rate from the game client to the server. Similar variations will occur at the server's side and impact the game state update transmission rates to the game clients.

This paper explores such a resource-constrained case and its effect on the popular team-based first-person shooter \textsc{Overwatch} on the PC. We recorded several game sessions and on this basis examine and model all lag-relevant components. These models are then used to refine the \gls{E2E} lag simulation of \cite{Metzger+2016}.