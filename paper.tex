%!TEX program = pdflatex
\documentclass[a4paper,conference]{IEEEtran}

\usepackage{ifluatex}
\ifluatex
  \usepackage{fontspec}
  \usepackage{polyglossia} % babel replacement for use with fontspec
  \setdefaultlanguage[variant=american]{english}
  \selectlanguage[variant=american]{english}
\else
  \usepackage[utf8]{inputenc}
  \usepackage[T1]{fontenc}
  \usepackage[american]{babel}
  \usepackage{textcomp}
  %\DeclareUnicodeCharacter{20AC}{\euro{}}
  %\usepackage{eurosym}
\fi

\usepackage[acronym, nomain, nowarn]{glossaries}
\loadglsentries{acronyms}
\usepackage[hyphens]{url}
%\usepackage[pdfhighlight=/O, hidelinks, unicode=true]{hyperref}
\usepackage{mathtools}
\usepackage{fixmath}
\usepackage{algorithm2e}
%\usepackage{subcaption}
\usepackage{siunitx}
%\DeclareSIUnit{\EUR}{\text{\texteuro}} % gulps up subsequent symbols, commented out
\DeclareSIUnit\year{yrs}
\sisetup{load-configurations = abbreviations,binary-units, per-mode=symbol}
\usepackage{csquotes}
\usepackage{booktabs}
\usepackage{tabu}
\usepackage{rotating}
\makeglossaries
\usepackage[url=false,doi=false,backend=biber,style=ieee,isbn=false,sorting=none,minnames=1,maxnames=2]{biblatex}
\addbibresource{literature.bib}
\usepackage[inline]{enumitem}

\usepackage[np]{numprint}
\npstyleenglish

\usepackage[bordercolor=white]{todonotes}
\usepackage{balance}


\usepackage{cleveref}
\crefformat{footnote}{#2\footnotemark[#1]#3}


\begin{document}

% \title{End-to-End Lag is Complicated: A Detailed Model for ``Overwatch''}
%\title{It's Complicated: An End-to-End Lag Model for the Multiplayer Shooter Overwatch}
\title{Exploring the Transmission Behaviour of Overwatch: The Source of Lag}

\author{
\IEEEauthorblockN{Florian Metzger}
\IEEEauthorblockA{Chair of Communication Networks\\
University of Würzburg, Germany\\
\texttt{florian.metzger@uni-wuerzburg.de}}
\and
\IEEEauthorblockN{Roman Heger}
%\IEEEauthorblockA{Chair of Modeling of Adaptive Systems\\
University of Duisburg-Essen, Germany\\
%\texttt{florian.metzger@uni-due.de}}
}

% Performance Modeling, User-centric Application Metrics, Interactivity, Video Games

\maketitle

% TODO: Perform proof-reading of the paper to correct minor spelling/grammar mistakes and (as stated in the previous part of the review) provide more details on distributions for readers with less mathematical knowledge. Minor improvements regarding readability of the paper, especially mathematical background, would be to provide either references or some overview information on distributions(Gamma, Normal,Cauchy).

% TODO: - Investigate auto-correlation and influence factors (additional delay, packet loss, smaller time-scales) on the lag.

% TODO: - The paper does the implicit assumption that the lag can be models as independent processes and there is no investigation of correlations/auto-correlation on smaller time scales. 
% TODO: - There is only one network configuration evaluated. Additional delay in the network or packet loss may also change the sending behavior of the client and/or server.
% TODO: - It is unclear how the models can be generalized. The measurements cover only one set of parameters.

% TODO: Proofread the paper, there are some minor punctuation issues and some double words and more. You should also consider to consult a native english speaker and let him rephrase the "worst" parts -> there is a lot of german grammar ... 

% TODO: I don't see any major weakness in this paper. It would be good to elaborate more on the investigation to help unfamiliar readers to understand the overall approach. Moreover, some background and related work would helpful to highlight the scientific contribution. However, this would very likely exceed the page limit.

% TODO: - The proposed methodology and model, as pointed out by the same Authors, is highly specific to the kind of game and to the deployment scenario (e.g., characteristics and resource availability at the player's PC). Which generality? Possibility to apply to different deployment scenarios? How?
% TODO: - There are no quantitative results showing that the proposed modeling is well fitting the experimental results about lag that can be directly measured in in-the-field experimentation. How to validate in-the-field the proposed model?

\input{acronyms.tex}
%!TEX root = paper.tex
%%%%%%%%%%%%%%%%%%%%%%%%%%%%%%%%%%%%%%%%%%%%%%%%%%%%%%%%%%%%%%%%%%%%%%%%%%%%%%%%
\begin{abstract}

This paper explores the full chain of lag contribution factors in a specific online multiplayer game, namely Overwatch: From the creation of input events, over the network, and back to displaying the results on the local screen. Together they result in the dreaded end-to-end lag, which has a direct impact on the subjective quality one experiences when playing video games. In its investigation, this paper reveals surprising effects in the game's networking behavior that are omitted when colloquially talking about, e.g. a \SI{60}{\hertz} update rate, but must be considered nonetheless. These insights, gained from examining network traces of Overwatch matches that were played on a realistic, resource constrained PC, can then be used to refine end-to-end lag simulation models and reach a better understanding of all responsible lag components.

\end{abstract}
%!TEX root = paper.tex
%%%%%%%%%%%%%%%%%%%%%%%%%%%%%%%%%%%%%%%%%%%%%%%%%%%%%%%%%%%%%%%%%%%%%%%%
%%%%%%%

\section{Introduction}

In theory, video games are simple: Retrieve an input, process it and update the game state, then display the results. But in praxis this gets a lot more complicated through the dynamic behavior of all involved factors. Especially when you add competitive multiplayer in to the mix and must synchronize the game states of the (authoritative) game server and each of the clients. Each of these components can then contribute to the \gls{E2E} lag, which will have an impact on each players' interactivity, and such on the perceived quality of the game as well \cite{Claypool:2006:LPA:1167838.1167860,7965676}. 

As to the extent of the impact of this lag on the perceived quality, many different factors can play a role, not just, as originally thought, the genre of the game in question \cite{mollertowards,Slivar:2016:CGQ:2910017.2910602}. In theory, the lag components of a video game can be measured \cite{7148095} or modeled in simulation \cite{Metzger+2016}. But in praxis the involved processes can get a lot more dynamic than stated in basic theory. The resource-constrained nature, where the video game can not maintain a stable frame rate, can play a major role. Most research in the past has simply assumed ideal conditions on this matter. Sufficient CPU and GPU resources to maintain, e.g., a frame rate of \SI{60}{\hertz}, as well as sufficient server resources to handle all of the users' input events. Yet, in reality this is often not the case. According to the Steam Hardware Survey\footnote{\url{https://store.steampowered.com/hwsurvey/}} the typical PC video game player tends not to have the latest or most powerful iteration of hardware components. But she still wants to turn up the game's settings to enjoy a better graphical fidelity, often at the cost of frame rate. This resource-constrainedness affects all associated components relevant to the \gls{E2E} lag, including the command send rate from the game client to the server. Similar variations will occur at the server's side and impact the game state update transmission rates to the game clients. It should be noted, that there are many anecdotes of competitive players doing the exact opposite and reduce graphical details to achieve both less visual distraction and better performance.

This paper explores such a resource-starved case and its effect on the popular team-based first-person shooter \textsc{Overwatch} on the PC. We recorded several game sessions and on this basis examine and model all factors that influence the lag. These models are then used to refine the \gls{E2E} lag simulation of \cite{Metzger+2016}, of which this work is a direct follow-up. By their very nature, the results are limited to this specific case and can not necessarily be easily generalized. And yet, the conditions in this experiment are not particularly rare. The game's ideal behavior is just that. Ideal, intended behavior in an environment where everything works perfectly. But due to the large variety in computer hardware and the fact that games often push the given hardware to their limits --- to produce good image quality at acceptable performance levels --- video games often run into resource limits. And this what makes the study of such corner cases nonetheless worthwhile.


% Anectdotally correct, but do we have any evidence, references on this?
% Or do we just argue that this is a common preference, as also targeted at by console manufactures/developers (aim for 30fps instead of something higher)
% Additionally: competitive players will often do the opposite to gain potential (and again, anectdotal) advantages. esp. e.g. observed in CSGO; advantages through both a better performance, but also through less visual complexity, less noise to distract

%!TEX root = paper.tex
%%%%%%%%%%%%%%%%%%%%%%%%%%%%%%%%%%%%%%%%%%%%%%%%%%%%%%%%%%%%%%%%%%%%%%%%%%%%%%%%
\section{Examining Individual Lag Components}
\label{sec:lagmodel}

% \begin{figure}[t]
% 	\centering
% 	\includegraphics[width=1.0\columnwidth]{images/e2e-lag-model.pdf}
% 	\caption{Basic \acrshort{E2E} lag model. Adapted from \cite{Metzger+2016}.}
% \label{fig:e2e-lag-model}
% \end{figure}

In any video game a multitude of cogs is working in unison to transform a player's inputs into meaningful actions inside the game world. Besides additional lag incurred through the input and output devices' hardware (i.e. in this case mouse/keyboard as well as the monitor) --- which can on their own still be significant\footnote{See, e.g. the investigation of keyboard latency at \url{https://danluu.com/keyboard-latency/}, in some cases exceeding \SI{50}{\milli\second}.}. The components (and the corresponding terminology) investigated here are outlined in \cite{Metzger+2016}, and are 
\begin{enumerate*}[label=(\alph*)]
	\item the generation of the input events,
	\item the command message send process (from the clients to the server),
	\item the network delay,
	\item the server's game state update rate (``tick rate''),
	\item the server's update send rate (server to clients),
	\item the client's frame rate (and additionally other lag-inducing factors at the client).
\end{enumerate*}
%
% --- all lag components examined in this work are depicted in Fig.~\ref{fig:e2e-lag-model}. 
%
For this experiment several regular matches (in the default mode with two teams of six players each) were played on a PC that can not maintain a stable frame rate of \SI{60}{\hertz} at all times. Packet traces of the game's UDP traffic were recorded on the same PC and are the basis for this examination. The code for models derived here as well as the modified simulation will be made available.


\subsection{Command Messages}

\begin{figure}[t]
	\centering
	\includegraphics[width=1.0\columnwidth]{images/command-density.pdf}
	\caption{Density plot of the command messages the client sends, showing a bimodal behavior instead of the expected single mode at \SI{16.7}{\milli\second}.}
\label{fig:command-density}
\end{figure}

These are the messages the game clients sends to the server containing all the player's inputs during that cycle. The developers of \textsc{Overwatch} advertise a tick rate (i.e. the frequency the server updates its game state) of \SI{60}{\hertz} (or an \gls{IAT} of about \SI{16.7}{\milli\second}. Therefore, one would assume that the command message rate behaves similarly. However, when looking at the density of the message \glspl{IAT} in Fig.~\ref{fig:command-density}, two modes are revealed, one at around \SI{24}{\milli\second} (or a send rate of \SI{42}{\hertz}) and the other one at \SI{12}{\milli\second} (\SI{83}{\hertz}). These modes are not distributed uniformly across the match but are rather separated into distinct transmission phases as the time series in Fig.~\ref{fig:command-timeseries} reveals. The duration of the match can be divided into two alternating phases. The first phase reveals a (weak) third mode centered around \SI{16.7}{\milli\second}, i.e. the expected behavior of the game to send its inputs at a rate of \SI{60}{\hertz} to the server. But in the second phase, the game sends with both the \glspl{IAT} observed in Fig.~\ref{fig:command-density} at a ratio of about four to one in favor of the \SI{12}{\milli\second} intervals. In order to model this behavior, the trace was split at those phases and each phase modeled separately. The \SI{60}{\hertz} phases were fitted to a Gamma distribution with $\Gamma(\alpha = 4.437922, \beta = 208.366)$, the second phases were mixed by a Gamma distribution $\Gamma(\alpha = 49.32119, \beta = 3970.154)$ for the \SI{83}{\hertz} portion and a Normal distribution $\mathcal{N}(\mu = 0.02359103, \sigma = 0.001369922)$ for \SI{42}{\hertz}). Finally, the phase lengths are giving by a set of two Normal distributions. $\mathcal{N}(\mu = 28.85714, \sigma = 1.216385)$ models the length of the unimodal \SI{60}{\hertz} phase, and $\mathcal{N}(\mu = 40.83333, \sigma = 2.054805)$ gives the length of the bimodal phase. A random sample generated with this model is depicted in the left panel of Fig.~\ref{fig:command-timeseries}.

\begin{figure}[t]
	\centering
	\includegraphics[width=1.0\columnwidth]{images/command-ts-annotated.pdf}
	\caption{Time series of the client-sent command messages both measured in the experiment as well as generated from the model. The different phases are highlighted.}
\label{fig:command-timeseries}
\end{figure}



	% 	- Wechselnde Phasen erkennbar bei langsamerem Rechner
	% 	- durchgehend gleich bleibendes Verhalten bei schnellerem Rechner (Ausnahme: Einbruch am Ende -> Ende des Spiels, PotG)
	% - Betrachtung der (zeitlichen) Verteilung der Zwischenankunftszeiten
	% 	- Zwei phasen deutlicher zu erkennen
	% 		- kürzere Phase mit einem difusen Peak (ca. 60 Hz)
	% 		- längere Phase mit zwei schärferen Peaks (ca. 42 Hz und 83 Hz)
	% 	- PDF zeigt vorallem die beiden Peaks der längeren Phasen (83Hz-Peak stärker ausgeprägt) - Peak der kurzen Phasen "verschwindet" in der Überlagerung


	% 	- Aufteilung des Datensatzes in die beiden Phasen
	% 	- Aufteilung der längeren Phasen nach den beiden Peaks
	% 	- Bestimmung der Phasenlängen
	% 	- Fitting der einzelnen Datensätze
	% 		- kurze Phasen: Gamma Verteilung
	% 		- lange Phasen - 83 Hz-Peak: Gamma Verteilung
	% 		- lange Phasen - 42 Hz-Peak: Normal Verteilung
	% 	- Komposition der Verteilungen
	% 		- Phasenlängen über Normalverteilung
	% 		- Wechsel in der langen Phase über Zufallsprozess gewichtet mit Wahrscheinlichkeit 0,8 für 83 Hz und 0,2 für 42 Hz

	% fit.short.phases.mean <- 28.85714
	% fit.short.phases.sd <- 1.216385

	% fit.long.phases.mean <- 40.83333
	% fit.long.phases.sd <- 2.054805

	% fit.short.shape <- 4.437922
	% fit.short.rate <- 208.366

	% fit.long.lower.shape <- 49.32119
	% fit.long.lower.rate <- 3970.154

	% fit.long.upper.mean <- 0.02359103
	% fit.long.upper.sd <- 0.001369922


\subsection{Game State Update Messages}

	\begin{figure}[t]
		\centering
		\includegraphics[width=1.0\columnwidth]{images/update-ts.pdf}
		\caption{Time series of the received game state update messages. The game phases and non-interactive ``kill cam'' events are clearly visible.}
	\label{fig:update-timeseries}
	\end{figure}
% TODO: mark phases in figure with vertical bars

	\begin{figure}[t]
		\centering
		\includegraphics[width=1.0\columnwidth]{images/update-density.pdf}
		\caption{Density plot of both the measured and the fitted game state update messages received by the client.}
	\label{fig:update-density}
	\end{figure}

	The other main lag component investigated here is the distribution of the game state update messages that the game server sends periodically to each connected game client, ideally one after every game simulation ``tick''. A time series of the client's packet reception \glspl{IAT} of one match is depicted in Fig.~\ref{fig:update-timeseries}. There are a few properties of note here, that clearly indicate that the server's messaging behavior depends on the game state even though generally targeting a rate of \SI{60}{\hertz}. The first block of very short \glspl{IAT} maps to the pick phase of \textsc{Overwatch}, where each player picks the hero (or class) she is going to play during this match. The next short phase (visible by less spread in the samples) is the warmup phase, where both teams can not yet leave their respective starting areas and thus no meaningful action can occur. Only after this phase the actual game starts and the spread of the \gls{IAT} increases, possibly hinting at increased load at the server side. Similar the final phase of less sample spread depicts the game after the match has ended and the players see match statistics as well as a replay of a noteworthy event in the match (the ``Play of the Game''). Every now and then the data exposes a quick burst of data in short intervals (i.e. the vertical bars in Fig.~\ref{fig:update-timeseries}). This directly correlates to the so-called ``kill cam''. When the player dies during the match, and before she respawns, she is shown a quick replay of the event from the perspective of the shooter. Thus, this replay data needs to be transferred to the player first. This feature can also be turned off, eliminating the burst of data alongside.	The reason for the grouping on the vertical axis into distinctly visible horizontal bars is not entirely clear yet, but is not necessarily present on each client PC. A quick check with the game running on another computer did not exhibit this behavior. Nonetheless, it is present here and certainly influences the \gls{E2E} lag and therefore should be considered in the model. Due to this oscillating behavior (cf. the density plot in Fig.~\ref{fig:update-density}) the update message sending process could be best modeled with a Cauchy distribution as base, modulated with a sine function. This yields
	%
	\begin{equation*}
		f_{rcv}(x) = \frac{0.7\sin(171.6 +3120x)^4 + 0.2}{0.9848153} f(x; x_0; \gamma)^{1.17} 
	\end{equation*}
	%
	for the density, with location $x_0 = 0.0155$ and scale $\gamma = 0.0011$ for the Cauchy distribution. It should be noted that this model is only applicable for the game's main phase and excludes the ``kill cam''-events as well. Those are either not relevant to the outcome of the match (warmup phase and post-match statistics) or are even entirely non-interactive for the player.






%	- Auffällige Peaks -> große Daten- / Paketmengen -> erklärt durch "Killcam-Videos" (synchronisierte Betrachtung von Video-Mitschnitten)
%	- Peaks verschwinden wenn "Killcam-Videos" ausgeschaltet werden (bis auf Peak am Ende -> "Play of the Game")

%	- zur Erzeugung der Submoden wird eine Cauchy-Vertielung über eine $sin^4$-Funktion moduliert
%	- Erzeugung der Verteilung durch Accept-Reject-Verfahren mit Hüllkurve $t(x) = Cauchy*3$ und $r(x) = Cauchy$



\subsection{Input Events}

Another component is the distribution of the actual input events that the player generates with her input devices. The original simulation just assumed an exponential distribution with a rate that might not be entirely realistic for an actual game. Therefore, for \textsc{Overwatch} the inputs were recorded and a rate parameter for the exponential distribution was derived as a compound of the button strokes and mouse movements ($\lambda = 102.1824)$.

		% - "mousotron"
		% - Messung von Key- und Mousestrokes (pro Zeit)
		% - Messung von zurückgelegter Strecke des Cursors (pro Zeit)
		% 		- Summe der Key- und Mousestrokes sowie der Anzahl an zurückgelegten Pixeln (bestimmt über zurückgelegte Strecke des Cursors * dots/cm des Monitors)
		% - bestimmung der Rate aus der o. g. Summe und der Messzeit
	 % Userinput: Exponentialverteilung mit oben bestimmter Rate

\subsection{Network Delay}

Finally, the network delay must also be considered. But due to the high availability of servers and the matchmaking that should place the player on to a server in the vicinity, the delay to the server should be low and stable. In a separate RTT measurement to the game server's IP address found in the investigated matches the RTT was about \SI{34}{\milli\second}.

	% constant
	% 	- 500 pings mit mittlerer Größe der gesendeten Pakete
	% 	- 500 pings mit mittlerer Größe der empfangenen Pakete


%!TEX root = paper.tex
%%%%%%%%%%%%%%%%%%%%%%%%%%%%%%%%%%%%%%%%%%%%%%%%%%%%%%%%%%%%%%%%%%%%%%%%%%%%%%%%
\section{Overwatch Lag Simulation}
\label{sec:simulation}

\begin{figure}
	\centering
	\includegraphics[width=1.0\columnwidth]{images/lagsim.pdf}
	\caption{\acrshort{E2E} lag results of a frame rate study using the \textsc{Overwatch} simulation.}
\label{fig:lagsim}
\end{figure}

% TODO: - Interpretation of Figure 5: what happens between 40 and 50 Hz (when the maximal E2E lag suddenly decreases from >1 s to >200 ms)? Why such an abrupt change?
% TODO: Why is there such a spread in the second phase? --- Answer we honestly do not know, but might be attributed to internal game behavior

Now that all client-observable parameters have been examined, a simulation study is performed on the effects of the game's frame rate. This is more or less the only parameter that the player can influence through the game's setting or through better performing hardware. But since, it is often tricky to maintain a stable frame rate in such resource constrained environments, the target frame rate is just assumed to be the mean of a normal distribution instead of a fixed value. This should account for the variations in the frame times (the time between two consecutive frames). Investigated here were frame rates between \SI{10}{\hertz} and \SI{120}{\hertz}, with the former an unrealistically low value, that nonetheless might occur in high stress situations when resource constraint, and the latter being on the upper end of what modern PC monitors can still display (current models reach roughly \SI{165}{\hertz}). All other simulation parameters are derived from the models in the previous section or left as-is in the base simulation. The results are depicted in Fig.~\ref{fig:lagsim}. Starting at about \SI{50}{\hertz} the \gls{E2E} lag reduction sees diminishing returns especially towards its outliers which can exceed a lag of over a second below \SI{50}{\hertz}. This correlates quite well with the generally accepted notion that (especially) online first person shooters require at least \SI{60}{\hertz} for an enjoyable experience. In praxis, only \SI{30}{\hertz}, \SI{60}{\hertz} or higher frame rates will be targeted due to otherwise occurring issues with the monitor's refresh rate and resulting screen tearing or uneven frame times.





% Server Delay (Bearbeitungszeit des Servers) übernommen als Normalverteilung aus Ursprünglicher Simulation
% - Frames: Normmalverteilung mit mittlerer Framerate von 60 Hz (evtl. Parameterstudie über mittelwert?)

%!TEX root = paper.tex
%%%%%%%%%%%%%%%%%%%%%%%%%%%%%%%%%%%%%%%%%%%%%%%%%%%%%%%%%%%%%%%%%%%%%%%%%%%%%%%%
\section{Conclusion}
\label{sec:conclusion}

Repeat again, observations for one specific computer / resource environmant, might be hard to generalize. But can still be indicative of general network issues in such environments, because such processes might be tied to other processes, e.g., the renderer, which can not keep their intended output rates.

This lag model for \textsc{Overwatch} is highly customized and unfortunately not a one-size-fits-all solution for every game. But it still gives much needed insights into the dynamics of video games and the deviations from the theoretical properties that are actually happening in praxis. With that being said, even the refined simulation clearly demonstrates the strong influence of the game's frame rate and input and messaging processes investigated here on the \gls{E2E} lag, despite operating under near-ideal network conditions.


\printbibliography

\balance

\end{document}
