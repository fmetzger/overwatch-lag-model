%!TEX root = paper.tex
%%%%%%%%%%%%%%%%%%%%%%%%%%%%%%%%%%%%%%%%%%%%%%%%%%%%%%%%%%%%%%%%%%%%%%%%%%%%%%%%
\begin{abstract}

This paper explores the full chain of lag contribution factors in a specific online multiplayer game, namely Overwatch: From the creation of input events, over the network, and back to displaying the results on the local screen. Together they result in the dreaded end-to-end lag, which has a direct impact on the subjective quality one experiences when playing video games. In its investigation, this paper reveals surprising effects in the game's networking behavior that are omitted when colloquially talking about, e.g. a \SI{60}{\hertz} update rate, but must be considered nonetheless. These insights, gained from examining network traces of Overwatch matches that were played on a realistic, resource constrained PC, can then be used to refine end-to-end lag simulation models and reach a better understanding of all responsible lag components.

\end{abstract}